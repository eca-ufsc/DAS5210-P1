\documentclass[11pt]{article}
\usepackage[brazilian]{babel}
\usepackage[utf8x]{inputenc}
\usepackage{enumitem}
\usepackage{pgfplots}
\usepackage{graphicx} 
\pgfplotsset{width=10cm, compat=1.9}
\usepackage[framed,numbered,autolinebreaks,useliterate]{mcode}
\setlength{\parindent}{4em}

\begin{document}
\begin{header_}
% ========== Edit your name here
\author{
José Fernando Rosa Ribeiro
}
\title{DAS5210 - Introdução ao Controle de Processos 
\newline
\newline
\large Prova 1
\date{\vspace{-5ex}}}
\maketitle
\setcounter{secnumdepth}{0}
\end{header_}

\section{Questão 1}
Monte o diagrama do sistema real (não-linear) em malha aberta usando o pacote Simulink, do
Matlab. Em seguida, simule este modelo, considerando o ponto de operação com i = 7:2mA. Para
tal simulação, varie em até 1 V o sinal u(t). Também simule o comportamento do sistema para
diferentes variações de tração de carga q(t), em até 5 N.m.

     \begin{figure}[!htb]
        \center{\includegraphics[width=\textwidth]
        {nao-linearizado.jpg}}
        \caption{\label{fig:my-label} Diagrama do sistema não-linearizado no Simulink.}
      \end{figure}
\section{Questão 2}
Monte o sistema linearizado completo em malha-aberta, no Simulink, do Matlab.
     \begin{figure}[!htb]
        \center{\includegraphics[width=\textwidth]
        {linearizado.jpg}}
        \caption{\label{fig:my-label} Diagrama do sistema linearizado.}
      \end{figure}
\section{Questão 3}
Usando Simulink, estude por simulação o comportamento deste sistema e compare o comportamento
com o do sistema não-linear nas proximidades do ponto de equilibrio estudado. Use os mesmos ensaios
do item 1.

\section{Questão 4}
\subsection{Itens a e b}
\begin{lstlisting}[caption={Código usado para controlar a planta},captionpos=b]
function [y1, y2]= fcn(u, m)

lower_bound = 5;
upper_bound = 15;
active = m;
if active
    y1 = 5;
else
    y1 = 1;
end

if u>=upper_bound
    active = 0;
elseif u<=lower_bound
    active = 1;
end
y2 = active;
\end{lstlisting}
\begin{figure}[!htb]
        \center{\includegraphics[width=\textwidth]
        {questao4/control_schema.jpg}}
        \caption{\label{fig:my-label} Esquema da planta no Simulink.}
      \end{figure}

\subsection{Item a}
\begin{figure}[!htb]
        \center{\includegraphics[width=\textwidth]
        {questao4/item_a/q_1.png}}
        \caption{\label{fig:my-label} Gráfico de u(t) (em vermelho) e $\omega(t)$ (em azul), variáveis manipulada e controlada da planta, respectivamente, para $Q(t) = 1 N.m$}
      \end{figure}
      
\subsection{Item b}
\begin{figure}[!htb]
        \center{\includegraphics[width=\textwidth]
        {questao4/item_b/q_5.png}}
        \caption{\label{fig:my-label} Gráfico de u(t) (em vermelho) e $\omega(t)$ (em azul), variáveis manipulada e controlada da planta, respectivamente, para $Q(t) = 5 N.m$}
      \end{figure}

\subsection{Item c}

\begin{figure}[!htb]
        \center{\includegraphics[width=\textwidth]
        {questao4/item_c/control_schema.png}}
        \caption{\label{fig:my-label} Gráfico de u(t) (em vermelho) e $\omega(t)$ (em azul), variáveis manipulada e controlada da planta, respectivamente, para $Q(t) = 5 N.m$}
      \end{figure}
    
\begin{figure}[!htb]
        \center{\includegraphics[width=\textwidth]
        {questao4/item_c/control_schema.png}}
        \caption{\label{fig:my-label} Gráfico de u(t) (em vermelho) e $\omega(t)$ (em azul), variáveis manipulada e controlada da planta, respectivamente, para $Q(t) = 5 N.m$}
      \end{figure}


\section{Questão 5}
Pretende-se "controlar” o sistema de velocidade do motor em malha-aberta, usando uma lei de
controle do tipo u(t) = KMAr(t), sendo r(t) uma referência de velocidade do tipo degrau. Ajuste o
ganho KMA e analise separamente as respostas temporais i(t) (considere variações do tipo degrau). É
possível, com esta estratégia, garantir o seguimento de referências de velocidade r(t) do tipo degrau
?
\section{Questão 6}
Compare a estratégia de controle acima com a estratégia On-Off e avalia as capacidades de ambas
em termos de seguimento de referência e rejeição de perturbações q(t) do tipo degrau.
\end{document}

